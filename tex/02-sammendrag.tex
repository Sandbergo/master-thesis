%!TEX root = ../Thesis.tex
\chapter*{\norwegianabstractname}
\addcontentsline{toc}{chapter}{\norwegianabstractname}
%
Dette prosjektet evaluerer Multi-Layer Perceptrons (MLP) for maskinlæringsassistert branching foreslått av Gupta et al. \cite{gupta2020hybrid} for mer effektiv løsning av blandede heltallsproblemer (\gls{MILP}). Effektive MILP-løsningsalgoritmer er viktige for optimering i sanntid i mange industrier, blant annet produksjon, logistikk, transport og energiproduksjon  \cite{junger2010years}. Flere sentrale forskere innen numerisk optimalisering og maskinlæring har vist interesse for å benytte maskinlæring i \gls{MILP}\cite{bengio2020machine,bertsimas2019online}. Spesielt variabelseleksjonsdelen av branchingstrategien har vist seg å være et felt der maskinlæringsmetoder kan gi lovende resultater \cite{khalil2020towards}. I 2019 ble pålitelige resultater med forbedring over de beste branchingstrategiene i åpen-kildekodeløsere vist \cite{gasse2019exact}, og i 2020 ble disse metodene utvidet til rent CPU-baserte modeller. Multi-layer perceptrons var inkludert i Gupta et al. \cite{gupta2020hybrid}, men effektiviteten ble ikke rapportert, og det er heller ikke avgjort hvorvidt konfigurasjonen er betydelig for den resulterende løsningstiden.

For å adressere dette blir tre multi-layer perceptron-konfigurasjoner trent gjennom imitasjonslæring av Strong Branching-algoritmen (\gls{SB}) på genererte \gls{MILP}-problemer, slik som i Gupta et al. \cite{gupta2020hybrid}. Modellene blir deretter inkorporert i optimaliseringsløseren \gls{SCIP} og evaluert på testproblemer av varierende størrelse. Alle modeller viser nær høyest effektivitet mot sammenlignbare algoritmer, hvor ettlags\gls{MLP}en viser den største forbedringen over Pseudo-cost- og Reliability Pseudo-cost-strategiene. Framtidig forskning bør videre verifisere disse resultatene på andre problemer, samt måle resultatene opp mot de beste proprietære løserene etter parameteroptimalisering \cite{hutter2010automated}. Analyse av bidraget fra de forskjellige inputparameterene til de lærte branchingstrategiene kan potensielt også føre til økt innsikt i variabelseleksjonsproblemet.

Koden for dette prosjektet er tilgjengelig på\\ \url{https://github.com/Sandbergo/learn2branch}
\clearpage
